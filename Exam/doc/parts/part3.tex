% !TeX spellcheck = en_GB
\section{Question 3}
% Question 3.1
\subsection{}
In figure \ref{code:3:1} my implementation of the transactional memory / multiverse version of \texttt{Cluster} can be seen. Most noticeable the \texttt{sumx}, \texttt{sumy} and \texttt{count} fields are now the transactional data types \texttt{TxnDouble} and \texttt{TxnInteger}, and instead of the synchronized keyword for locking, the atomic keyword is used with a lambda for those parts of the methods which should be handled as transactions. Again note that the most important part the exam assignment is that \texttt{addToMean} is done atomically to ensure that \texttt{KMeans2Stm}. Mean does not change in the assignment part and therefore it does not pose a threat that it is accessed concurrently. The rest of the methods are not called on the same \texttt{Cluster} concurrently.

\begin{figure}
\begin{lstlisting}
static class Cluster extends ClusterBase {
    private Point mean;
    private final TxnDouble sumx, sumy;
    private final TxnInteger count;
    public Cluster(Point mean) {
        sumx = newTxnDouble();
        sumy = newTxnDouble();
        count = newTxnInteger();
        this.mean = mean;
    }
    public void addToMean(Point p) {
        atomic (() -> {
            sumx.set(sumx.get() + p.x);
            sumy.set(sumy.get() + p.y);
            count.increment();
        });
    }
    public boolean computeNewMean() {
        Point oldMean = this.mean;
        atomic( () -> {
            this.mean = new Point(sumx.get()/count.get(), sumy.get()/count.get());
        });
        return oldMean.almostEquals(this.mean);
    }
    public void resetMean() {
        atomic( () -> {
            sumx.set(0.0);
            sumy.set(0.0);
            count.set(0);
        });
    }
    @Override
    public Point getMean() { return mean; }
}
\end{lstlisting}
\caption{Thread safety using transaction in KMeans2Stm}
\label{code:3:1}
\end{figure}

% Question 3.2
\subsection{}
Here the running time and iterations of \texttt{KMeans2Stm} can be seen:
\begin{lstlisting}
# OS:   Windows 10; 10.0; amd64
# JVM:  Oracle Corporation; 1.8.0_101
# CPU:  Intel64 Family 6 Model 94 Stepping 3, GenuineIntel; 8 "cores"
# Date: 2017-01-10T19:42:09+0100
...
Used 108 iterations
class kmean.KMeans2Stm Real time:     3.214
...
Used 108 iterations
class kmean.KMeans2Stm Real time:     2.924
...
Used 108 iterations
class kmean.KMeans2Stm Real time:     3.287
\end{lstlisting}

% Question 3.3
\subsection{}
These are the first 5 means of \texttt{KMeans2Stm}
\begin{lstlisting}
mean = (87,96523339545510, 68,02660923585381)
mean = (10,68258565222153, 47,90993881688515)
mean = (17,94392765121715, 57,92167357590438)
mean = (48,03811144575212, 28,07238725661516)
mean = (78,02580276649083, 68,03938084899143)
\end{lstlisting}

% Question 3.4
\subsection{}
Since the \texttt{KMeans2Stm} logic is exactly the same as in \texttt{KMeans2P} the same concurrency problems would occur if I removed the transactions code. The program would "never" terminate because each modification to the \texttt{sumx}, \texttt{sumy} and \texttt{count} fields could be overwritten due to a race-condition which would be different each time.
