% !TeX spellcheck = en_GB
\section{Question 6}
% Question 6.1
\subsection{}
In the figures \ref{code:6:1-1}, \ref{code:6:1-2}, \ref{code:6:1-3}, \ref{code:6:1-4} and \ref{code:6:1-5} my Java AKKA implementation of the specification given in \texttt{start.erl} is shown.

\begin{figure}
\begin{lstlisting}
public static void main(String[] args)
{
    final ActorSystem system = ActorSystem.create("OddEven");
    final ActorRef dispatcher = system.actorOf(Props.create(DispatcherActor.class), "dispather");
    final ActorRef odd = system.actorOf(Props.create(WorkerActor.class), "odd");
    final ActorRef even = system.actorOf(Props.create(WorkerActor.class), "even");
    final ActorRef collector = system.actorOf(Props.create(CollectorActor.class), "collector");
    dispatcher.tell(new InitMessage(odd, even, collector), ActorRef.noSender());
    for(int i = 1; i<=10; i++) {
        dispatcher.tell(new NumMessage(i), ActorRef.noSender());
    }
    try {
        System.out.println("Press to Terminate");
        System.in.read();
    } catch (Exception e) {
        e.printStackTrace();
    } finally {
        system.shutdown();
    }
}
\end{lstlisting}
\caption{The start implementation of the system specified by start.erl}
\label{code:6:1-1}
\end{figure}


\begin{figure}
\begin{lstlisting}
public class DispatcherActor extends UntypedActor {
    private ActorRef odd, even;
    @Override
    public void onReceive(Object message) throws Throwable {
        if(message instanceof InitMessage) {
            odd = ((InitMessage) message).getOdd();
            even = ((InitMessage) message).getEven();
            odd.tell(message, ActorRef.noSender());
            even.tell(message, ActorRef.noSender());
        }
        else if(message instanceof NumMessage) {
            int value = ((NumMessage) message).getNumber();
            if(value % 2 == 0) {
                even.tell(message, ActorRef.noSender());
            } else {
                odd.tell(message, ActorRef.noSender());
            }
        }
    }
}
\end{lstlisting}
\caption{The dispatcher actor implementation specified by start.erl}
\label{code:6:1-2}
\end{figure}


\begin{figure}
\begin{lstlisting}
public class WorkerActor extends UntypedActor {
    private ActorRef collector;
    @Override
    public void onReceive(Object message) throws Throwable {
        if(message instanceof InitMessage) {
            collector = ((InitMessage) message).getCollector();
        } else if(message instanceof  NumMessage) {
            int value = ((NumMessage) message).getNumber();
            collector.tell(new NumMessage(value*value), ActorRef.noSender());
        }
    }
}
\end{lstlisting}
\caption{The worker actor implementation specified by start.erl}
\label{code:6:1-3}
\end{figure}



\begin{figure}
\begin{lstlisting}
public class CollectorActor extends UntypedActor {
    @Override
    public void onReceive(Object message) throws Throwable {
        if(message instanceof  NumMessage) {
            System.out.println(((NumMessage) message).getNumber());
        }
    }
}
\end{lstlisting}
\caption{TThe collector actor implementation specified by start.erl}
\label{code:6:1-4}
\end{figure}



\begin{figure}
\begin{lstlisting}
public class InitMessage implements Serializable {
    private final ActorRef odd;
    private final ActorRef even;
    private final ActorRef collector;
    public InitMessage(ActorRef odd, ActorRef even, ActorRef collector) {
        this.odd = odd;
        this.even = even;
        this.collector = collector;
    }
    public ActorRef getOdd() { return odd; }
    public ActorRef getEven() { return even; }
    public ActorRef getCollector() { return collector; }
}

public class NumMessage implements Serializable {
    private final int number;
    public NumMessage(int number) { this.number = number; }
    public int getNumber() { return number; }
}
\end{lstlisting}
\caption{The init and num message implementation specified by start.erl}
\label{code:6:1-5}
\end{figure}

% Question 6.2
\subsection{}
The system works and produces the following output:
\begin{lstlisting}
4
1
9
25
16
36
49
64
81
100
\end{lstlisting}
This is of course not in order, but as stated in the exam assignment that is expected.